\documentclass{article}
\usepackage[T2A]{fontenc}
\usepackage[cp1251]{inputenc}
\usepackage{amsthm}
\usepackage{amsmath}
\usepackage{amssymb}
\usepackage{amsfonts}
\usepackage{mathrsfs}
\usepackage[12pt]{extsizes}
\usepackage{fancyvrb}
\usepackage{indentfirst}
\usepackage[ left=2cm, right=2cm, top=2cm, bottom=2cm, headsep=0.2cm,
  footskip=0.6cm, bindingoffset=0cm ]{geometry}
\usepackage[english,russian]{babel}


\begin{document}
\section*{Вариант 26}
Уравнение математической модели:
\begin{equation}
  q_{0} (q_{0}^{0} + q_{\text{т}}^{0} + Kp_{x}^{0}) + q(q_{0}^{\text{т}} + q_{\text{т}}^{\text{т}} + Kp_{x}^{\text{т}}) + p_{0} (q_{0}^{\text{тр}} + q_{\text{т}}^{\text{тр}} + Kp_{x}^{\text{тр}}) = 1.
  \label{eq:first}
\end{equation}

По аналогии с предыдущим вариантом модели условные вероятности $q_0((l + 1)/l),
  q((l + 1)/l), p_x((l + 1)/l)$ правильного приема нулевого, токового, стирания,
соответственно, для $(l + 1)$"=го символа запишутся в виде:

\begin{equation}
  \left.
  \begin{array}{lll}
    q_0((l + 1)/l)          & = & q_0q_0^0 + qq_0^{\text{т}} + Kp_xq_0^x;                           \\
    q_{\text{т}}((l + 1)/l) & = & q_0q_{\text{т}}^0 +qq_{\text{т}}^{\text{т}} + Kp_xq_{\text{т}}^x; \\
    p_0((l + 1)/l)          & = & q_0p_{\text{тр}}^0 +qp_{\text{тр}}^{\text{т}} + Kp_xp_x^x.
  \end{array}
  \right\}
  \label{eq:second}
\end{equation}

Решая систему уравнений (\ref{eq:second}), можно получить формулы для определения безусловных ве-
роятностей $q_0, q, q_x.$

\end{document}
