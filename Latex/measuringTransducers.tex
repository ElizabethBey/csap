\documentclass{article}
\usepackage[T2A]{fontenc}
\usepackage[cp1251]{inputenc}
\usepackage{amsthm}
\usepackage{amsmath}
\usepackage{amssymb}
\usepackage{amsfonts}
\usepackage{mathrsfs}
\usepackage[12pt]{extsizes}
\usepackage{fancyvrb}
\usepackage{indentfirst}
\usepackage[ left=2cm, right=2cm, top=2cm, bottom=2cm, headsep=0.2cm,
  footskip=0.6cm, bindingoffset=0cm ]{geometry}
\usepackage[english,russian]{babel}
\usepackage{graphicx}
\graphicspath{{pictures/}} \DeclareGraphicsExtensions{.pdf,.png,.jpg}

\begin{document}

\begin{center}\textbf{\LARGE Измерительные преобразователи}\end{center}
\vspace{0.5cm}

Измерительный преобразователь "--- специальное устройство, которое преобразует
величину неэлектрического характера в электросигнал, а также наоборот. К
преобразователям также относятся приборы, переводящие измеряемый параметр в иную
величину, которая будет удобной для исследования, преобразования, в том числе
сохранения и передачи. Эти приборы необходимы во многих сферах, поэтому они
получили значительное распространение.
\vspace{1cm}
\begin{center}\textbf{\Large Классификация}\end{center}
\vspace{0.5cm}

По характеру преобразования:
\begin{description}
  \item["---] аналоговый измерительный преобразователь "--- измерительный
    преобразователь, преобразующий одну аналоговую величину (аналоговый
    измерительный сигнал) в другую аналоговую величину (измерительный сигнал);
  \item["---] аналого"=цифровой измерительный преобразователь "--- измерительный
    преобразователь, предназначенный для преобразования аналогового
    измерительного сигнала в цифровой код;
  \item["---] цифро"=аналоговый измерительный преобразователь "--- измерительный
    преобразователь, предназначенный для преобразования числового кода в
    аналоговую величину.
\end{description}

\vspace{1cm}
По месту в измерительной цепи:
\begin{description}
  \item["---] первичный измерительный преобразователь "--- измерительный
    преобразователь, на который непосредственно воздействует измеряемая
    физическая величина. Первичный измерительный преобразователь является первым
    преобразователем в измерительной цепи измерительного прибора;
  \item["---] датчик "--- конструктивно обособленный первичный преобразователь,
    от которого поступают измерительные сигналы;
  \item["---] детектор "--- датчик в области измерений ионизирующих излучений;
  \item["---] промежуточный измерительный преобразователь "--- измерительный
    преобразователь, занимающий место в измерительной цепи после первичного
    преобразователя.
\end{description}

\vspace{1cm}
По другим признакам:
\begin{description}
  \item["---] передающий измерительный преобразователь "--- измерительный
    преобразователь, предназначенный для дистанционной передачи сигнала
    измерительной информации;
  \item["---] масштабный измерительный преобразователь "--- измерительный
    преобразователь, предназначенный для изменения размера величины или
    измерительного сигнала в заданное число раз.
\end{description}

\vspace{1cm}
По принципу действия:
\begin{description}
  \item["---] параметрические или пассивные датчики, в которых изменение
    контролируемой величины сопровождается изменением сопротивления датчика
    (активного, индуктивного, емкостного). При этом наличие постороннего
    источника энергии является обязательным условием работы параметрического
    датчика;
  \item["---] генераторные или активные датчики, в которых изменение
    контролируемой величины Х сопровождается изменением ЭДС на выходе датчика,
    возникновение ЭДС может происходить за счет термоэлектричества, пьезоэффекта
    и т.д.
\end{description}

\vspace{1cm}
К основным характеристикам первичных измерительных преобразователей относятся:
\begin{itemize}
  \item \textbf{Входная величина}, воспринимаемая и преобразуемая датчиком;
  \item \textbf{Выходная величина}, используемая для передачи информации;
  \item \textbf{Статическая характеристика датчика}. Для каждого измерительного
        преобразователя можно установить связь между выходной и входной
        величинами.
\end{itemize}

\vspace{1cm}
\begin{center}\textbf{\Large Устройство}\end{center}
\vspace{0.5cm}

Имеется достаточно обширное разнообразие измерительных устройств. Однако вне
зависимости от их видового разнообразия у всех у них имеется первичный
измерительный преобразователь, который и проводит измерение величины. Как раз
его, в конечном счете, и необходимо измерить, но величина на выходе должна быть
уже в электрическом виде.

Измеряемая величина воздействует на датчик, который находится в месте измерений
и выполняет функции первичного преобразователя. Далее находится промежуточный
преобразователь, который переводит сигнал в удобную для восприятия величину. Они
могут выполнять различные задачи:

\begin{description}
  \item["---] масштабно-временное преобразование;
  \item["---] цифро-аналоговое преобразование;
  \item["---] масштабное преобразование;
  \item["---] изменение величины;
  \item["---] функциональное преобразование и так далее.
\end{description}

Однако следует учитывать, что в цепи могут находиться сразу несколько первичных
преобразователей. Рассмотрим ризистивные и тензорезистивные датчики.

\vspace{1cm}
\begin{center}\textbf{\Large Резистивные ИП}\end{center}
\vspace{0.5cm}

Потенциометрические ИП преобразуют механические перемещения в изменения
сопротивления реостата. По назначению датчики бывают линейных и угловых
перемещений. Потенциометрический датчик представляет собой реостат, включённый
по схеме потенциометра. При перемещении подвижного контакта под воздействием
контролируемой величины Х происходит изменение сопротивления датчика. В
зависимости от закона изменения сопротивления различают линейные и
функциональные потенциометры, а в зависимости от схемы включения полярные и
реверсивные.

Введем обозначения $R_0$ полное сопротивление потенциометра, $R_x$ сопротивление
при заданном положении движка, $R_{\text{н}}$ сопротивление нагрузки.
\begin{figure}[h]
  \center{\includegraphics[]{image1.png}}
  \caption{Потенциометрические датчики: а) полярный, б) реверсивный}
  \label{fig:image}
\end{figure}

Достоинства потенциометрических датчиков: простота конструкции, возможность
получения достаточно прямолинейной характеристики, стабильность характеристик,
значительная величина выходного сигнала.

Недостатки: пониженная надежность, износ, контактное сопротивление, относительно
большие перемещения и малая скорость движка, дискретность.

\vspace{1cm}
\begin{center}\textbf{\Large Тензорезистивные датчики}\end{center}
\vspace{0.5cm}
Для изменения усилий и деформаций в деталях и конструкциях различных устройств
применяются тензометрические или тензорезистивные датчики. Тензоэффект "---
изменение активного сопротивления проводников при механической деформации
материала. Величина тензоэффекта зависит от ориентации силы и вида материала.

Тензочувствительность $K_\text{T}$ - это отношение величины относительного
изменения его сопротивления к относительному изменению линейного размера
проволоки:
\[
  K_T = \frac{dR}{dl} = \frac{d\rho}{dl} + (1 + 2\mu)
\];
где $R$ "--- сопротивление провода, $l$ "--- начальная длина деформируемого
участка провода, $1 + 2\mu$ "--- характеризует собой изменение геометрических
размеров, $\mu$ "--- коэффициент Пуассона, $m = \frac{d\rho}{dl}$ "---
коэффициент изменения удельного сопротивления материала с изменением его
геометрических размеров.

Типы тензорезистивных датчиков: проволочные, фольговые, пленочные и
полупроводниковые (тензолиты).

\begin{center}\textbf{\Large Применение}\end{center}

Измерительный преобразователь находит широчайшее применение. Такие устройства
применяют на многих производствах, лабораториях и даже в быту. Это могут быть
сложные приборы, которые собирают многочисленную информацию с датчиков или же
простые устройства в виде домашних кухонных весов. Можно назвать следующие
области:
\begin{description}
  \item["---] металлургическая промышленность;
  \item["---] химическая и газовая промышленность;
  \item["---] научные и лабораторные установки;
  \item["---] медицина;
  \item["---] геология;
  \item["---] атомная промышленность;
  \item["---] энергетика.
\end{description}

На любом производстве, где требуется наблюдение или регулирование
технологического процесса, не обойтись без преобразователя. Такие
преобразователи часто используются в специальных измерительных приборах, которые
применяются для обработки сигналов:
\begin{description}
  \item["---] портативные измерительные приборы, к примеру, для получения
    показателей параметров воды или грунта;
  \item["---] щитовые приборы, которые имеются практически в каждом здании;
  \item["---] регистраторы и самописцы. Это сложнейшие приборы, которые
    отслеживают происходящие вокруг изменения и сохраняют все в памяти;
  \item["---] цифровые преобразователи;
  \item["---] весовые дозаторы, конвейерные и кухонные весы и так далее.
\end{description}
\end{document}
