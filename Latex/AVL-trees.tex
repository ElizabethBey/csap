\documentclass{beamer}
\usetheme[]{Berlin}
\usepackage[T2A]{fontenc}
\usepackage[utf8]{inputenc}
\usepackage[english,russian]{babel}
\usepackage{graphicx}
\DeclareGraphicsExtensions{.png,.jpg}

% Данные для заголовочного слайда
\title{AVL-деревья}
\author{Бей Елизавета}
\institute{Саратовский национальный исследовательский государственный университет имени Н. Г. Чернышевского}
\date{}
\begin{document}

\frame{\titlepage}

\begin{frame}{Создатели AVL-дерева}
    AVL"=деревья "--- наверное, самый первый вид сбалансированных двоичных
    деревьев поиска, придуманных еще в 1962 году советскими учеными
    Адельсон"=Вельским и Ландисом.
\end{frame}

\begin{frame}{Определение}
    Рассмотрим модифицированный класс деревьев, обладающих
    всеми преимуществами бинарных деревьев поиска и никогда не вырождающихся.
    Они называются сбалансированными или AVL-деревьями. Под сбалансированностью
    будем понимать то, что для каждого узла дерева высоты обоих его поддеревьев
    различаются не более чем на 1.
\end{frame}

\begin{frame}{Пример 1}
    \begin{figure}
        \includegraphics[width=0.8\textwidth, height=0.65\textheight]{image12}
        \caption{1}
    \end{figure}

\end{frame}

\begin{frame}{Пример 2}
    \begin{figure}
        \includegraphics[width=0.8\textwidth, height=0.65\textheight]{image2}
        \caption{2}
    \end{figure}
\end{frame}

\begin{frame}{Высота}
    Максимальная высота AVL"=дерева при заданном числе узлов:
    \begin{equation}
        h \leqslant \frac{\log_{2}{(n + a)}}{b} - c \leqslant 1.45\log_{2}{(n + 2)}
    \end{equation}
    где: \[a = \frac{1 + \frac{3}{\sqrt{5} }}{2} \approx 1,17082039324994;\]
    \[ b = \log_{2}{(\frac{1 + \sqrt{5}}{2})} \approx 0,694241913630617;\]
    \[c = 2 - \frac{\log_{2}{5}}{2b} \approx 0,327724061815446.\]
\end{frame}

\begin{frame}[fragile]{Структура узлов}
    Будем представлять узлы AVL"=дерева следующей структурой:
    \begin{verbatim}
struct node //структура для представления узлов дерева
{
	    int key;
	    unsigned char height;
	    node* left;
	    node* right;
	    node(int k) { key = k; left = right = 0; height = 1; }
};
    \end{verbatim}
\end{frame}

\begin{frame}[fragile]{Вспомогательные функции}
    \begin{verbatim}
unsigned char height(node* p)
{
        return p ? p->height : 0;
}

int bfactor(node* p)
{
	    return height(p->right) - height(p->left);
}
    \end{verbatim}
\end{frame}

\begin{frame}[fragile]{Вспомогательные функции}
    \begin{verbatim}
void fixheight(node* p)
{
	    unsigned char hl = height(p->left);
	    unsigned char hr = height(p->right);
	    p->height = (hl > hr ? hl : hr) + 1;
}
\end{verbatim}
\end{frame}

\begin{frame}{Правый поворот}
    \begin{figure}
        \includegraphics[width=0.8\textwidth, height=0.55\textheight]{rightrot.png}
        \caption{правый поворот}
    \end{figure}
\end{frame}

\begin{frame}[fragile]{Балансировка узлов: правый поворот}
    \begin{verbatim}
node* rotateright(node* p) // правый поворот вокруг p
{
    node* q = p->left;
    p->left = q->right;
    q->right = p;
    fixheight(p);
    fixheight(q);
    return q;
}
    \end{verbatim}
\end{frame}

\begin{frame}{Левый поворот}
    \begin{figure}
        \includegraphics[width=0.8\textwidth, height=0.55\textheight]{leftrot.png}
        \caption{левый поворот}
    \end{figure}
\end{frame}

\begin{frame}[fragile]{Балансировка узлов: левый поворот}
    \begin{verbatim}
node* rotateleft(node* q) // левый поворот вокруг q
{
    node* p = q->right;
    q->right = p->left;
    p->left = q;
    fixheight(q);
    fixheight(p);
    return p;
}
    \end{verbatim}
\end{frame}

\begin{frame}{Простой поворот}
    \begin{figure}
        \includegraphics[width=0.8\textwidth, height=0.55\textheight]{bigrotate1.png}
        \caption{простой поворот}
    \end{figure}
\end{frame}

\begin{frame}{Большой поворот}
    \begin{figure}
        \includegraphics[width=\textwidth, height=0.5\textheight]{bigrotate2.png}
        \caption{большой поворот}
    \end{figure}
\end{frame}

\begin{frame}[fragile]{Балансировка}
    \begin{small}
        \begin{verbatim}
node* balance(node* p){ // балансировка узла p
    fixheight(p);
    if(bfactor(p)== 2) {
        if(bfactor(p->right) < 0)
            p->right = rotateright(p->right);
        return rotateleft(p);
    }
    if(bfactor(p)== -2) {
        if( bfactor(p->left) > 0  )
            p->left = rotateleft(p->left);
        return rotateright(p);
    }
    return p; // балансировка не нужна
}
    \end{verbatim}
    \end{small}
\end{frame}

\begin{frame}[fragile]{Балансировка}
    \begin{verbatim}
node* insert(node* p, int k) 
{//вставка ключа k в дерево с корнем p
    if(!p) return new node(k);
    if(k < p->key)
        p->left = insert(p->left,k);
    else
        p->right = insert(p->right,k);
    return balance(p);
}
    \end{verbatim}
\end{frame}

\begin{frame}{Удаление}
    \begin{figure}
        \includegraphics[width=0.8\textwidth, height=0.5\textheight]{delete.png}
        \caption{удаление}
    \end{figure}
\end{frame}

\begin{frame}[fragile]{Вспомогательные функции}
    \begin{verbatim}
node* findmin(node* p) 
{//поиск узла с минимальным ключом в дереве p 
    return p->left ? findmin(p->left) : p;
}

node* removemin(node* p) 
{//удаление узла с минимальным ключом из дерева p
	    if(p->left==0)
		        return p->right;
	    p->left = removemin(p->left);
	    return balance(p);
}
    \end{verbatim}
\end{frame}

\begin{frame}[fragile]{Функция удаления}
    \begin{small}
        \begin{verbatim}
node* remove(node* p, int k) {
    if(!p) return 0;
    if(k < p->key)
        p->left = remove(p->left,k);
    else if(k > p->key)
        p->right = remove(p->right,k);	
        else {
            node* q = p->left;
            node* r = p->right;
            delete p;
            if( !r ) return q;
            node* min = findmin(r);
            min->right = removemin(r);
            min->left = q;
            return balance(min);
        }
    return balance(p);
}
    \end{verbatim}
    \end{small}
\end{frame}

\begin{frame}
    \begin{beamercolorbox}[ht=7ex, dp=4ex, center, shadow=false, rounded=true]{title in head/foot}
        \centerline{\LARGE{Спасибо за внимание!}}
    \end{beamercolorbox}
\end{frame}
\end{document}
